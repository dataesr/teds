% Options for packages loaded elsewhere
\PassOptionsToPackage{unicode}{hyperref}
\PassOptionsToPackage{hyphens}{url}
%
\documentclass[
]{article}
\usepackage{amsmath,amssymb}
\usepackage{lmodern}
\usepackage{ifxetex,ifluatex}
\ifnum 0\ifxetex 1\fi\ifluatex 1\fi=0 % if pdftex
  \usepackage[T1]{fontenc}
  \usepackage[utf8]{inputenc}
  \usepackage{textcomp} % provide euro and other symbols
\else % if luatex or xetex
  \usepackage{unicode-math}
  \defaultfontfeatures{Scale=MatchLowercase}
  \defaultfontfeatures[\rmfamily]{Ligatures=TeX,Scale=1}
\fi
% Use upquote if available, for straight quotes in verbatim environments
\IfFileExists{upquote.sty}{\usepackage{upquote}}{}
\IfFileExists{microtype.sty}{% use microtype if available
  \usepackage[]{microtype}
  \UseMicrotypeSet[protrusion]{basicmath} % disable protrusion for tt fonts
}{}
\makeatletter
\@ifundefined{KOMAClassName}{% if non-KOMA class
  \IfFileExists{parskip.sty}{%
    \usepackage{parskip}
  }{% else
    \setlength{\parindent}{0pt}
    \setlength{\parskip}{6pt plus 2pt minus 1pt}}
}{% if KOMA class
  \KOMAoptions{parskip=half}}
\makeatother
\usepackage{xcolor}
\IfFileExists{xurl.sty}{\usepackage{xurl}}{} % add URL line breaks if available
\IfFileExists{bookmark.sty}{\usepackage{bookmark}}{\usepackage{hyperref}}
\hypersetup{
  pdftitle={Mapping scientific communities at scale},
  pdfkeywords={French publications, Machine
learning, scanR, OpenAlex, Biblioglutton, Elasticsearch},
  hidelinks,
  pdfcreator={LaTeX via pandoc}}
\urlstyle{same} % disable monospaced font for URLs
\usepackage[left=3cm, right=3cm, top=3cm, bottom=3cm]{geometry}
\usepackage{color}
\usepackage{fancyvrb}
\newcommand{\VerbBar}{|}
\newcommand{\VERB}{\Verb[commandchars=\\\{\}]}
\DefineVerbatimEnvironment{Highlighting}{Verbatim}{commandchars=\\\{\}}
% Add ',fontsize=\small' for more characters per line
\newenvironment{Shaded}{}{}
\newcommand{\AlertTok}[1]{\textcolor[rgb]{1.00,0.00,0.00}{\textbf{#1}}}
\newcommand{\AnnotationTok}[1]{\textcolor[rgb]{0.38,0.63,0.69}{\textbf{\textit{#1}}}}
\newcommand{\AttributeTok}[1]{\textcolor[rgb]{0.49,0.56,0.16}{#1}}
\newcommand{\BaseNTok}[1]{\textcolor[rgb]{0.25,0.63,0.44}{#1}}
\newcommand{\BuiltInTok}[1]{#1}
\newcommand{\CharTok}[1]{\textcolor[rgb]{0.25,0.44,0.63}{#1}}
\newcommand{\CommentTok}[1]{\textcolor[rgb]{0.38,0.63,0.69}{\textit{#1}}}
\newcommand{\CommentVarTok}[1]{\textcolor[rgb]{0.38,0.63,0.69}{\textbf{\textit{#1}}}}
\newcommand{\ConstantTok}[1]{\textcolor[rgb]{0.53,0.00,0.00}{#1}}
\newcommand{\ControlFlowTok}[1]{\textcolor[rgb]{0.00,0.44,0.13}{\textbf{#1}}}
\newcommand{\DataTypeTok}[1]{\textcolor[rgb]{0.56,0.13,0.00}{#1}}
\newcommand{\DecValTok}[1]{\textcolor[rgb]{0.25,0.63,0.44}{#1}}
\newcommand{\DocumentationTok}[1]{\textcolor[rgb]{0.73,0.13,0.13}{\textit{#1}}}
\newcommand{\ErrorTok}[1]{\textcolor[rgb]{1.00,0.00,0.00}{\textbf{#1}}}
\newcommand{\ExtensionTok}[1]{#1}
\newcommand{\FloatTok}[1]{\textcolor[rgb]{0.25,0.63,0.44}{#1}}
\newcommand{\FunctionTok}[1]{\textcolor[rgb]{0.02,0.16,0.49}{#1}}
\newcommand{\ImportTok}[1]{#1}
\newcommand{\InformationTok}[1]{\textcolor[rgb]{0.38,0.63,0.69}{\textbf{\textit{#1}}}}
\newcommand{\KeywordTok}[1]{\textcolor[rgb]{0.00,0.44,0.13}{\textbf{#1}}}
\newcommand{\NormalTok}[1]{#1}
\newcommand{\OperatorTok}[1]{\textcolor[rgb]{0.40,0.40,0.40}{#1}}
\newcommand{\OtherTok}[1]{\textcolor[rgb]{0.00,0.44,0.13}{#1}}
\newcommand{\PreprocessorTok}[1]{\textcolor[rgb]{0.74,0.48,0.00}{#1}}
\newcommand{\RegionMarkerTok}[1]{#1}
\newcommand{\SpecialCharTok}[1]{\textcolor[rgb]{0.25,0.44,0.63}{#1}}
\newcommand{\SpecialStringTok}[1]{\textcolor[rgb]{0.73,0.40,0.53}{#1}}
\newcommand{\StringTok}[1]{\textcolor[rgb]{0.25,0.44,0.63}{#1}}
\newcommand{\VariableTok}[1]{\textcolor[rgb]{0.10,0.09,0.49}{#1}}
\newcommand{\VerbatimStringTok}[1]{\textcolor[rgb]{0.25,0.44,0.63}{#1}}
\newcommand{\WarningTok}[1]{\textcolor[rgb]{0.38,0.63,0.69}{\textbf{\textit{#1}}}}
\setlength{\emergencystretch}{3em} % prevent overfull lines
\providecommand{\tightlist}{%
  \setlength{\itemsep}{0pt}\setlength{\parskip}{0pt}}
\setcounter{secnumdepth}{-\maxdimen} % remove section numbering
\ifluatex
  \usepackage{selnolig}  % disable illegal ligatures
\fi
\newlength{\cslhangindent}
\setlength{\cslhangindent}{1.5em}
\newlength{\csllabelwidth}
\setlength{\csllabelwidth}{3em}
\newenvironment{CSLReferences}[2] % #1 hanging-ident, #2 entry spacing
 {% don't indent paragraphs
  \setlength{\parindent}{0pt}
  % turn on hanging indent if param 1 is 1
  \ifodd #1 \everypar{\setlength{\hangindent}{\cslhangindent}}\ignorespaces\fi
  % set entry spacing
  \ifnum #2 > 0
  \setlength{\parskip}{#2\baselineskip}
  \fi
 }%
 {}
\usepackage{calc}
\newcommand{\CSLBlock}[1]{#1\hfill\break}
\newcommand{\CSLLeftMargin}[1]{\parbox[t]{\csllabelwidth}{#1}}
\newcommand{\CSLRightInline}[1]{\parbox[t]{\linewidth - \csllabelwidth}{#1}\break}
\newcommand{\CSLIndent}[1]{\hspace{\cslhangindent}#1}
% for compatibility with pandoc 2.10
\newenvironment{cslreferences}%
  {\setlength{\parindent}{0pt}%
  \everypar{\setlength{\hangindent}{\cslhangindent}}\ignorespaces}%
  {\par}

\title{Mapping scientific communities at scale}
\usepackage{authblk}
\author[%
  1%
  ]{%
  Hafsa Aallat%
  %
  %
}
\affil[1]{French Ministry of Higher Education and Research, Paris,
France}
\date{February 2025}

\makeatletter
\def\@maketitle{%
  \newpage \null \vskip 2em
  \begin {center}%
    \let \footnote \thanks
         {\LARGE \@title \par}%
         \vskip 1.5em%
                {\large \lineskip .5em%
                  \begin {tabular}[t]{c}%
                    \@author
                  \end {tabular}\par}%
                                                \vskip 1em{\large \@date}%
  \end {center}%
  \par
  \vskip 1.5em}
\makeatother

\begin{document}
\maketitle

\textbf{Keywords}: open access, open science, open data, open source

\hypertarget{motivation}{%
\section{1. Motivation}\label{motivation}}

\hypertarget{presentation-of-ipcc-and-ipbes-working-groups-and-dates}{%
\subsection{1.1 Presentation of IPCC and IPBES: Working Groups and
dates}\label{presentation-of-ipcc-and-ipbes-working-groups-and-dates}}

\textbf{The IPCC (Intergovernmental Panel on Climate Change)} assesses
scientific information on climate change, providing reports to guide
policymakers. It has three working groups sees as three main themes :

\begin{itemize}
\tightlist
\item
  Working Group I (WGI) focuses on the \textbf{physical science} of
  climate change.
\item
  Working Group II (WGII) examines climate change impacts,
  \textbf{adaptation}, and vulnerabilities.
\item
  Working Group III (WGIII) addresses climate change \textbf{mitigation}
  strategies.
\end{itemize}

The Sixth Assessment Report (AR6) was released in stages between 2021
and 2022.

\textbf{The IPBES (Intergovernmental Science-Policy Platform on
Biodiversity and Ecosystem Services)}, established in 2012, assesses
biodiversity and ecosystem services. It produces thematic and regional
assessments, with the \textbf{Global Assessment Report (2019)}
highlighting biodiversity loss and the need for urgent action.

Both platforms provide crucial scientific assessments that inform global
climate and biodiversity policies.

\hypertarget{limits-of-the-french-court-of-audit-study}{%
\subsection{1.2 Limits of the French Court of Audit
study}\label{limits-of-the-french-court-of-audit-study}}

In 2023, the French Court of Audit conducted a study on France's
scientific output related to environmental transition. After hearings
with the Directorate General for Research and Innovation (DGRI) and
research operators, the Court analyzed the bibliography cited in the
sixth IPCC report. The study found that French publications are the most
cited in the physical sciences of climate change, highlighting the
global impact of French research in this field.

However, this evaluation has important limitations. The IPCC
bibliography is based on high-impact publications often from top
journals, making it quite selective. This selection prioritizes more
visible and well-known works, leaving out other important research that
may not be as visible but still in the same themes as IPCC report. While
this reflects France's scientific excellence, it does not fully
represent the diversity of French scientific contributions to ecological
transition.

\hypertarget{how-can-we-explore-and-recognize-french-publications-related-to-the-same-topics-as-ipcc-report-from-a-global-point-of-view}{%
\subsection{1.3 How can we explore and recognize french publications
related to the same topics as IPCC report from a global point of view
?}\label{how-can-we-explore-and-recognize-french-publications-related-to-the-same-topics-as-ipcc-report-from-a-global-point-of-view}}

To fill this gap, we propose using a larger dataset, such as scanR.
\textbf{ScanR has a significantly higher coverage} of publications with
at least one French affiliation compared to other sources, contributing
92\% to the overall aggregated corpus. This is much higher than
databases like Scopus (67\%), WoS (58\%), or PubMed (29\%), making ScanR
a more comprehensive tool for capturing French scientific publications
(Chaignon and Egret 2022). Unlike the IPCC's restricted approach, ScanR
includes publications with at least one French affiliation, showing a
larger view of research. This could allow us to capture a more diverse
range of topics related to climate change physical science, adaptation
and mitigation.

Initially, we will replicate the Court of Audit analysis of the IPCC
bibliography to identify the main themes and their proportion of French
contributions. Then, we will expand our study to know the top
institutions, labs, regions, and researchers that provide solutions to
the challenges of environemental transition in France, based on IPCC
bibliography. In a second time, we will create a model that can
recognize a publication about IPCC similar topics, and apply the model
to scanR publications. At the same time, we will conduct a similar
analysis for the IPBES bibliography, following the same approach to
identify the French contributions, and exploring less visible but
valuable research related to biodiversity and ecosystem services.

\hypertarget{ipcc-and-ipbes-bibliography-analysis-and-model}{%
\subsubsection{2. IPCC and IPBES Bibliography Analysis and
Model}\label{ipcc-and-ipbes-bibliography-analysis-and-model}}

We propose a method to analyze the bibliographies of IPCC and IPBES
reports.

\hypertarget{data-collection-and-cleaning}{%
\paragraph{2.1 Data Collection and
Cleaning}\label{data-collection-and-cleaning}}

For each report, we collect the references:

\begin{itemize}
\tightlist
\item
  For IPCC report, we collect citations in .bib format for each chapter
  of each working group (n.d.a).
\item
  For IPBES report, we gather all citations via Zotero (n.d.b).
\end{itemize}

Once the data is collected, we clean the DOI (Digital Object Identifier)
of each publication. The DOI should follow a specific format starting
with `10.'. Any publication without a valid DOI is not considered.

\hypertarget{data-enrichment}{%
\paragraph{2.2 Data Enrichment}\label{data-enrichment}}

After cleaning, the data contains features such as DOI, title, and main
author. However, we still lack information such as institutions,
researchers, countries, and topics associated with each publication. To
fill in the gap, we enrich the data by importing additional features
from OpenAlex for each publication with a valid DOI. These features
include: countries, year, topics, title, author names, institutions,
RORs (Research Organization Registry) and journals.

OpenAlex is an international open-access database that provides metadata
on research papers, authors, journals, and institutions. It aims to make
academic information more accessible and supports data analysis and
knowledge discovery in various fields. OpenAlex is a valuable tool for
researchers and educators. We use the Api to import the features.

Next, we use the Biblioglutton Python library to fill in missing DOIs
based on the title and main author. We also verify that the year
retrieved from OpenAlex matches the year in the original dataset.

\hypertarget{data-storage-and-visualization}{%
\subsection{2.3 Data storage and
visualization}\label{data-storage-and-visualization}}

Once the data is enriched with openAlex features, we edit the data and
push them on a cluster elastic-search. As the exemple, for one
publication:

\begin{Shaded}
\begin{Highlighting}[]
\FunctionTok{\{}
  \DataTypeTok{"doi"}\FunctionTok{:} \StringTok{"10.1126/science.aaw6974"}\FunctionTok{,}
  \DataTypeTok{"year"}\FunctionTok{:} \StringTok{"2018"}\FunctionTok{,}
  \DataTypeTok{"title"}\FunctionTok{:} \StringTok{"Impacts of 1.5 °C global warming on natural and human systems"}\FunctionTok{,}
  \DataTypeTok{"rors"}\FunctionTok{:} \OtherTok{[}
    \OtherTok{[}\StringTok{"https://ror.org/00rqy9422"}\OtherTok{,} \StringTok{"AU"}\OtherTok{],}
    \OtherTok{[}\StringTok{"https://ror.org/03ztgj037"}\OtherTok{,} \StringTok{"DE"}\OtherTok{],}
    \OtherTok{[}\StringTok{"https://ror.org/03fkc8c64"}\OtherTok{,} \StringTok{"JM"}\OtherTok{],}
    \OtherTok{[}\StringTok{"https://ror.org/03ztgj037"}\OtherTok{,} \StringTok{"DE"}\OtherTok{],}
    \OtherTok{[}\StringTok{"https://ror.org/04jr1s763"}\OtherTok{,} \StringTok{"IT"}\OtherTok{],}
    \OtherTok{[}\StringTok{"https://ror.org/01ryk1543"}\OtherTok{,} \StringTok{"GB"}\OtherTok{],}
    \OtherTok{[}\StringTok{"https://ror.org/05wwcw481"}\OtherTok{,} \StringTok{"GB"}\OtherTok{],}
    \OtherTok{[}\StringTok{"https://ror.org/05k07f122"}\OtherTok{,} \StringTok{"AR"}\OtherTok{],}
    \OtherTok{[}\StringTok{"https://ror.org/03cqe8w59"}\OtherTok{,} \StringTok{"AR"}\OtherTok{],}
    \OtherTok{[}\StringTok{"https://ror.org/0081fs513"}\OtherTok{,} \StringTok{"AR"}\OtherTok{],}
    \OtherTok{[}\StringTok{"https://ror.org/05sbt2524"}\OtherTok{,} \StringTok{"FR"}\OtherTok{],}
    \OtherTok{[}\StringTok{"https://ror.org/02feahw73"}\OtherTok{,} \StringTok{"FR"}\OtherTok{],}
    \OtherTok{[}\StringTok{"https://ror.org/02rx3b187"}\OtherTok{,} \StringTok{"FR"}\OtherTok{],}
    \OtherTok{[}\StringTok{"https://ror.org/01wwcfa26"}\OtherTok{,} \StringTok{"FR"}\OtherTok{],}
    \OtherTok{[}\StringTok{"https://ror.org/05q3vnk25"}\OtherTok{,} \StringTok{"FR"}\OtherTok{],}
    \OtherTok{[}\StringTok{"https://ror.org/03vmsb260"}\OtherTok{,} \StringTok{"JP"}\OtherTok{],}
    \OtherTok{[}\StringTok{"https://ror.org/02j4mf075"}\OtherTok{,} \StringTok{"ID"}\OtherTok{],}
    \OtherTok{[}\StringTok{"https://ror.org/00cvxb145"}\OtherTok{,} \StringTok{"US"}\OtherTok{],}
    \OtherTok{[}\StringTok{"https://ror.org/03rp50x72"}\OtherTok{,} \StringTok{"ZA"}\OtherTok{],}
    \OtherTok{[}\StringTok{"https://ror.org/04ex24z53"}\OtherTok{,} \StringTok{"FR"}\OtherTok{],}
    \OtherTok{[}\StringTok{"https://ror.org/035xkbk20"}\OtherTok{,} \StringTok{"FR"}\OtherTok{],}
    \OtherTok{[}\StringTok{"https://ror.org/01pa4h393"}\OtherTok{,} \StringTok{"FR"}\OtherTok{],}
    \OtherTok{[}\StringTok{"https://ror.org/05q3vnk25"}\OtherTok{,} \StringTok{"FR"}\OtherTok{],}
    \OtherTok{[}\StringTok{"https://ror.org/02feahw73"}\OtherTok{,} \StringTok{"FR"}\OtherTok{],}
    \OtherTok{[}\StringTok{"https://ror.org/02hw5fp67"}\OtherTok{,} \StringTok{"JP"}\OtherTok{],}
    \OtherTok{[}\StringTok{"https://ror.org/00ae7jd04"}\OtherTok{,} \StringTok{"US"}\OtherTok{],}
    \OtherTok{[}\StringTok{"https://ror.org/013meh722"}\OtherTok{,} \StringTok{"GB"}\OtherTok{],}
    \OtherTok{[}\StringTok{"https://ror.org/0524sp257"}\OtherTok{,} \StringTok{"GB"}\OtherTok{],}
    \OtherTok{[}\StringTok{"https://ror.org/032e6b942"}\OtherTok{,} \StringTok{"DE"}\OtherTok{],}
    \OtherTok{[}\StringTok{"https://ror.org/05a28rw58"}\OtherTok{,} \StringTok{"CH"}\OtherTok{],}
    \OtherTok{[}\StringTok{"https://ror.org/02yr08r26"}\OtherTok{,} \StringTok{"DE"}\OtherTok{],}
    \OtherTok{[}\StringTok{"https://ror.org/01c8qhb70"}\OtherTok{,} \StringTok{"BS"}\OtherTok{],}
    \OtherTok{[}\StringTok{"https://ror.org/040tfy969"}\OtherTok{,} \StringTok{"GB"}\OtherTok{],}
    \OtherTok{[}\StringTok{"https://ror.org/026k5mg93"}\OtherTok{,} \StringTok{"GB"}\OtherTok{],}
    \OtherTok{[}\StringTok{"https://ror.org/034b53w38"}\OtherTok{,} \StringTok{"CN"}\OtherTok{]}
  \OtherTok{]}\FunctionTok{,}
  \DataTypeTok{"ipcc"}\FunctionTok{:} \OtherTok{[}
    \FunctionTok{\{} \DataTypeTok{"name"}\FunctionTok{:} \StringTok{"wg1\_chap\_01"}\FunctionTok{,} \DataTypeTok{"wg"}\FunctionTok{:} \StringTok{"1"}\FunctionTok{,} \DataTypeTok{"chap"}\FunctionTok{:} \DecValTok{1} \FunctionTok{\}}\OtherTok{,}
    \FunctionTok{\{} \DataTypeTok{"name"}\FunctionTok{:} \StringTok{"wg2\_chap\_01"}\FunctionTok{,} \DataTypeTok{"wg"}\FunctionTok{:} \StringTok{"2"}\FunctionTok{,} \DataTypeTok{"chap"}\FunctionTok{:} \DecValTok{1} \FunctionTok{\}}\OtherTok{,}
    \FunctionTok{\{} \DataTypeTok{"name"}\FunctionTok{:} \StringTok{"wg2\_chap\_02"}\FunctionTok{,} \DataTypeTok{"wg"}\FunctionTok{:} \StringTok{"2"}\FunctionTok{,} \DataTypeTok{"chap"}\FunctionTok{:} \DecValTok{2} \FunctionTok{\}}\OtherTok{,}
    \FunctionTok{\{} \DataTypeTok{"name"}\FunctionTok{:} \StringTok{"wg2\_chap\_02"}\FunctionTok{,} \DataTypeTok{"wg"}\FunctionTok{:} \StringTok{"2"}\FunctionTok{,} \DataTypeTok{"chap"}\FunctionTok{:} \DecValTok{2} \FunctionTok{\}}\OtherTok{,}
    \FunctionTok{\{} \DataTypeTok{"name"}\FunctionTok{:} \StringTok{"wg2\_chap\_04"}\FunctionTok{,} \DataTypeTok{"wg"}\FunctionTok{:} \StringTok{"2"}\FunctionTok{,} \DataTypeTok{"chap"}\FunctionTok{:} \DecValTok{4} \FunctionTok{\}}\OtherTok{,}
    \FunctionTok{\{} \DataTypeTok{"name"}\FunctionTok{:} \StringTok{"wg2\_chap\_07"}\FunctionTok{,} \DataTypeTok{"wg"}\FunctionTok{:} \StringTok{"2"}\FunctionTok{,} \DataTypeTok{"chap"}\FunctionTok{:} \DecValTok{7} \FunctionTok{\}}\OtherTok{,}
    \FunctionTok{\{} \DataTypeTok{"name"}\FunctionTok{:} \StringTok{"wg2\_chap\_08"}\FunctionTok{,} \DataTypeTok{"wg"}\FunctionTok{:} \StringTok{"2"}\FunctionTok{,} \DataTypeTok{"chap"}\FunctionTok{:} \DecValTok{8} \FunctionTok{\}}\OtherTok{,}
    \FunctionTok{\{} \DataTypeTok{"name"}\FunctionTok{:} \StringTok{"wg2\_chap\_12"}\FunctionTok{,} \DataTypeTok{"wg"}\FunctionTok{:} \StringTok{"2"}\FunctionTok{,} \DataTypeTok{"chap"}\FunctionTok{:} \DecValTok{12} \FunctionTok{\}}\OtherTok{,}
    \FunctionTok{\{} \DataTypeTok{"name"}\FunctionTok{:} \StringTok{"wg2\_chap\_13"}\FunctionTok{,} \DataTypeTok{"wg"}\FunctionTok{:} \StringTok{"2"}\FunctionTok{,} \DataTypeTok{"chap"}\FunctionTok{:} \DecValTok{13} \FunctionTok{\}}\OtherTok{,}
    \FunctionTok{\{} \DataTypeTok{"name"}\FunctionTok{:} \StringTok{"wg2\_chap\_14"}\FunctionTok{,} \DataTypeTok{"wg"}\FunctionTok{:} \StringTok{"2"}\FunctionTok{,} \DataTypeTok{"chap"}\FunctionTok{:} \DecValTok{14} \FunctionTok{\}}\OtherTok{,}
    \FunctionTok{\{} \DataTypeTok{"name"}\FunctionTok{:} \StringTok{"wg2\_chap\_15"}\FunctionTok{,} \DataTypeTok{"wg"}\FunctionTok{:} \StringTok{"2"}\FunctionTok{,} \DataTypeTok{"chap"}\FunctionTok{:} \DecValTok{15} \FunctionTok{\}}\OtherTok{,}
    \FunctionTok{\{} \DataTypeTok{"name"}\FunctionTok{:} \StringTok{"wg2\_chap\_15"}\FunctionTok{,} \DataTypeTok{"wg"}\FunctionTok{:} \StringTok{"2"}\FunctionTok{,} \DataTypeTok{"chap"}\FunctionTok{:} \DecValTok{15} \FunctionTok{\}}\OtherTok{,}
    \FunctionTok{\{} \DataTypeTok{"name"}\FunctionTok{:} \StringTok{"wg2\_chap\_16"}\FunctionTok{,} \DataTypeTok{"wg"}\FunctionTok{:} \StringTok{"2"}\FunctionTok{,} \DataTypeTok{"chap"}\FunctionTok{:} \DecValTok{16} \FunctionTok{\}}\OtherTok{,}
    \FunctionTok{\{} \DataTypeTok{"name"}\FunctionTok{:} \StringTok{"wg2\_cross\_chap\_1"}\FunctionTok{,} \DataTypeTok{"wg"}\FunctionTok{:} \StringTok{"2\_cross"}\FunctionTok{,} \DataTypeTok{"chap"}\FunctionTok{:} \DecValTok{1} \FunctionTok{\}}\OtherTok{,}
    \FunctionTok{\{} \DataTypeTok{"name"}\FunctionTok{:} \StringTok{"wg2\_cross\_chap\_4"}\FunctionTok{,} \DataTypeTok{"wg"}\FunctionTok{:} \StringTok{"2\_cross"}\FunctionTok{,} \DataTypeTok{"chap"}\FunctionTok{:} \DecValTok{4} \FunctionTok{\}}\OtherTok{,}
    \FunctionTok{\{} \DataTypeTok{"name"}\FunctionTok{:} \StringTok{"wg2\_cross\_chap\_4"}\FunctionTok{,} \DataTypeTok{"wg"}\FunctionTok{:} \StringTok{"2\_cross"}\FunctionTok{,} \DataTypeTok{"chap"}\FunctionTok{:} \DecValTok{4} \FunctionTok{\}}\OtherTok{,}
    \FunctionTok{\{} \DataTypeTok{"name"}\FunctionTok{:} \StringTok{"wg3\_chap\_01"}\FunctionTok{,} \DataTypeTok{"wg"}\FunctionTok{:} \StringTok{"3"}\FunctionTok{,} \DataTypeTok{"chap"}\FunctionTok{:} \DecValTok{1} \FunctionTok{\}}\OtherTok{,}
    \FunctionTok{\{} \DataTypeTok{"name"}\FunctionTok{:} \StringTok{"wg3\_chap\_04"}\FunctionTok{,} \DataTypeTok{"wg"}\FunctionTok{:} \StringTok{"3"}\FunctionTok{,} \DataTypeTok{"chap"}\FunctionTok{:} \DecValTok{4} \FunctionTok{\}}
  \OtherTok{]}\FunctionTok{,}
  \DataTypeTok{"authors\_name"}\FunctionTok{:} \OtherTok{[}
    \OtherTok{[}\StringTok{"Ove Hoegh‐Guldberg"}\OtherTok{,} \OtherTok{[}\StringTok{"AU"}\OtherTok{]],}
    \OtherTok{[}\StringTok{"Daniela Jacob"}\OtherTok{,} \OtherTok{[}\StringTok{"DE"}\OtherTok{]],}
    \OtherTok{[}\StringTok{"Michael A. Taylor"}\OtherTok{,} \OtherTok{[}\StringTok{"JM"}\OtherTok{]],}
    \OtherTok{[}\StringTok{"Tania Guillén Bolaños"}\OtherTok{,} \OtherTok{[}\StringTok{"DE"}\OtherTok{]],}
    \OtherTok{[}\StringTok{"Marco Bindi"}\OtherTok{,} \OtherTok{[}\StringTok{"IT"}\OtherTok{]],}
    \OtherTok{[}\StringTok{"Sally Brown"}\OtherTok{,} \OtherTok{[}\StringTok{"GB"}\OtherTok{]],}
    \OtherTok{[}\StringTok{"Inés Angela Camilloni"}\OtherTok{,} \OtherTok{[}\StringTok{"AR"}\OtherTok{]],}
    \OtherTok{[}\StringTok{"Arona Diedhiou"}\OtherTok{,} \OtherTok{[}\StringTok{"FR"}\OtherTok{]],}
    \OtherTok{[}\StringTok{"Riyanti Djalante"}\OtherTok{,} \OtherTok{[}\StringTok{"ID"}\OtherTok{,} \StringTok{"JP"}\OtherTok{]],}
    \OtherTok{[}\StringTok{"Kristie L. Ebi"}\OtherTok{,} \OtherTok{[}\StringTok{"US"}\OtherTok{]],}
    \OtherTok{[}\StringTok{"François Engelbrecht"}\OtherTok{,} \OtherTok{[}\StringTok{"ZA"}\OtherTok{]],}
    \OtherTok{[}\StringTok{"Joël Guiot"}\OtherTok{,} \OtherTok{[}\StringTok{"FR"}\OtherTok{]],}
    \OtherTok{[}\StringTok{"Yasuaki Hijioka"}\OtherTok{,} \OtherTok{[}\StringTok{"JP"}\OtherTok{]],}
    \OtherTok{[}\StringTok{"Shagun Mehrotra"}\OtherTok{,} \OtherTok{[}\StringTok{"US"}\OtherTok{]],}
    \OtherTok{[}\StringTok{"Chris Hope"}\OtherTok{,} \OtherTok{[}\StringTok{"GB"}\OtherTok{]],}
    \OtherTok{[}\StringTok{"Antony J. Payne"}\OtherTok{,} \OtherTok{[}\StringTok{"GB"}\OtherTok{]],}
    \OtherTok{[}\StringTok{"Hans‐Otto Pörtner"}\OtherTok{,} \OtherTok{[}\StringTok{"DE"}\OtherTok{]],}
    \OtherTok{[}\StringTok{"Sonia I. Seneviratne"}\OtherTok{,} \OtherTok{[}\StringTok{"CH"}\OtherTok{]],}
    \OtherTok{[}\StringTok{"Adelle Thomas"}\OtherTok{,} \OtherTok{[}\StringTok{"BS"}\OtherTok{,} \StringTok{"DE"}\OtherTok{]],}
    \OtherTok{[}\StringTok{"Rachel Warren"}\OtherTok{,} \OtherTok{[}\StringTok{"GB"}\OtherTok{]],}
    \OtherTok{[}\StringTok{"Guangsheng Zhou"}\OtherTok{,} \OtherTok{[}\StringTok{"CN"}\OtherTok{]]}
  \OtherTok{]}\FunctionTok{,}
  \DataTypeTok{"institutions\_names"}\FunctionTok{:} \OtherTok{[}
    \OtherTok{[}\StringTok{"University of Queensland"}\OtherTok{,} \StringTok{"AU"}\OtherTok{],}
    \OtherTok{[}\StringTok{"German Climate Computing Centre"}\OtherTok{,} \StringTok{"DE"}\OtherTok{],}
    \OtherTok{[}\StringTok{"University of the West Indies"}\OtherTok{,} \StringTok{"JM"}\OtherTok{],}
    \OtherTok{[}\StringTok{"German Climate Computing Centre"}\OtherTok{,} \StringTok{"DE"}\OtherTok{],}
    \OtherTok{[}\StringTok{"University of Florence"}\OtherTok{,} \StringTok{"IT"}\OtherTok{],}
    \OtherTok{[}\StringTok{"University of Southampton"}\OtherTok{,} \StringTok{"GB"}\OtherTok{],}
    \OtherTok{[}\StringTok{"Bournemouth University"}\OtherTok{,} \StringTok{"GB"}\OtherTok{],}
    \OtherTok{[}\StringTok{"Instituto Franco{-}Argentino sobre Estudios de Clima y sus Impactos"}\OtherTok{,} \StringTok{"AR"}\OtherTok{],}
    \OtherTok{[}\StringTok{"Consejo Nacional de Investigaciones Científicas y Técnicas"}\OtherTok{,} \StringTok{"AR"}\OtherTok{],}
    \OtherTok{[}\StringTok{"University of Buenos Aires"}\OtherTok{,} \StringTok{"AR"}\OtherTok{],}
    \OtherTok{[}\StringTok{"Grenoble Institute of Technology"}\OtherTok{,} \StringTok{"FR"}\OtherTok{],}
    \OtherTok{[}\StringTok{"French National Centre for Scientific Research"}\OtherTok{,} \StringTok{"FR"}\OtherTok{],}
    \OtherTok{[}\StringTok{"Université Grenoble Alpes"}\OtherTok{,} \StringTok{"FR"}\OtherTok{],}
    \OtherTok{[}\StringTok{"Institute of Environmental Geosciences"}\OtherTok{,} \StringTok{"FR"}\OtherTok{],}
    \OtherTok{[}\StringTok{"Institut de Recherche pour le Développement"}\OtherTok{,} \StringTok{"FR"}\OtherTok{],}
    \OtherTok{[}
      \StringTok{"United Nations University Institute for the Advanced Study of Sustainability"}\OtherTok{,}
      \StringTok{"JP"}
    \OtherTok{],}
    \OtherTok{[}\StringTok{"Haluoleo University"}\OtherTok{,} \StringTok{"ID"}\OtherTok{],}
    \OtherTok{[}\StringTok{"University of Washington"}\OtherTok{,} \StringTok{"US"}\OtherTok{],}
    \OtherTok{[}\StringTok{"University of the Witwatersrand"}\OtherTok{,} \StringTok{"ZA"}\OtherTok{],}
    \OtherTok{[}\StringTok{"Collège de France"}\OtherTok{,} \StringTok{"FR"}\OtherTok{],}
    \OtherTok{[}\StringTok{"Aix{-}Marseille University"}\OtherTok{,} \StringTok{"FR"}\OtherTok{],}
    \OtherTok{[}\StringTok{"Centre for Research and Teaching in Environmental Geoscience"}\OtherTok{,} \StringTok{"FR"}\OtherTok{],}
    \OtherTok{[}\StringTok{"Institut de Recherche pour le Développement"}\OtherTok{,} \StringTok{"FR"}\OtherTok{],}
    \OtherTok{[}\StringTok{"French National Centre for Scientific Research"}\OtherTok{,} \StringTok{"FR"}\OtherTok{],}
    \OtherTok{[}\StringTok{"National Institute for Environmental Studies"}\OtherTok{,} \StringTok{"JP"}\OtherTok{],}
    \OtherTok{[}\StringTok{"World Bank"}\OtherTok{,} \StringTok{"US"}\OtherTok{],}
    \OtherTok{[}\StringTok{"University of Cambridge"}\OtherTok{,} \StringTok{"GB"}\OtherTok{],}
    \OtherTok{[}\StringTok{"University of Bristol"}\OtherTok{,} \StringTok{"GB"}\OtherTok{],}
    \OtherTok{[}
      \StringTok{"Alfred{-}Wegener{-}Institut Helmholtz{-}Zentrum für Polar{-} und Meeresforschung"}\OtherTok{,}
      \StringTok{"DE"}
    \OtherTok{],}
    \OtherTok{[}\StringTok{"ETH Zurich"}\OtherTok{,} \StringTok{"CH"}\OtherTok{],}
    \OtherTok{[}\StringTok{"Climate Analytics"}\OtherTok{,} \StringTok{"DE"}\OtherTok{],}
    \OtherTok{[}\StringTok{"College of The Bahamas"}\OtherTok{,} \StringTok{"BS"}\OtherTok{],}
    \OtherTok{[}\StringTok{"Tyndall Centre"}\OtherTok{,} \StringTok{"GB"}\OtherTok{],}
    \OtherTok{[}\StringTok{"University of East Anglia"}\OtherTok{,} \StringTok{"GB"}\OtherTok{],}
    \OtherTok{[}\StringTok{"Chinese Academy of Meteorological Sciences"}\OtherTok{,} \StringTok{"CN"}\OtherTok{]}
  \OtherTok{]}\FunctionTok{,}
  \DataTypeTok{"countries"}\FunctionTok{:} \OtherTok{[}
    \StringTok{"CHN"}\OtherTok{,}
    \StringTok{"GBR"}\OtherTok{,}
    \StringTok{"FRA"}\OtherTok{,}
    \StringTok{"ITA"}\OtherTok{,}
    \StringTok{"AUS"}\OtherTok{,}
    \StringTok{"JAM"}\OtherTok{,}
    \StringTok{"DEU"}\OtherTok{,}
    \StringTok{"JPN"}\OtherTok{,}
    \StringTok{"ZAF"}\OtherTok{,}
    \StringTok{"USA"}\OtherTok{,}
    \StringTok{"BHS"}\OtherTok{,}
    \StringTok{"CHE"}\OtherTok{,}
    \StringTok{"IDN"}\OtherTok{,}
    \StringTok{"ARG"}
  \OtherTok{]}\FunctionTok{,}
  \DataTypeTok{"ipbes"}\FunctionTok{:} \OtherTok{[}\FunctionTok{\{} \DataTypeTok{"chapter"}\FunctionTok{:} \StringTok{"4"} \FunctionTok{\}}\OtherTok{]}\FunctionTok{,}
  \DataTypeTok{"topics"}\FunctionTok{:} \OtherTok{[}
    \StringTok{"Impact of Climate Change on Human Migration"}\OtherTok{,}
    \StringTok{"Geoengineering and Climate Ethics"}\OtherTok{,}
    \StringTok{"Economic Implications of Climate Change Policies"}
  \OtherTok{]}
\FunctionTok{\}}
\end{Highlighting}
\end{Shaded}

After that we used Highcharts, a graphic tool to visualize the graph.

\hypertarget{create-a-database}{%
\subsection{2.4 Create a database}\label{create-a-database}}

\hypertarget{train-the-model}{%
\subsection{2.5 Train the model}\label{train-the-model}}

\hypertarget{results}{%
\section{3. Results}\label{results}}

\hypertarget{section}{%
\subsection{3.1}\label{section}}

\hypertarget{custom-perimeter}{%
\subsection{3.2 Custom perimeter}\label{custom-perimeter}}

scanR offers this mapping tool for the entire indexed corpus, but it is
also possible to adapt the tool to a restricted perimeter, at the user's
discretion. For example, an institution or laboratory can define its own
corpus (based on a list of publications) and a mapping tool dedicated to
this perimeter is automatically created. Technically, elasticsearch
queries are the same, with just an additional filter to query only the
publications within the perimeter. The tool can be embedded in any
website using an iframe. It's the same principle as the local barometer.
This approach eliminates the need for automatic alignment of
affiliations, which remains a highly complex task. Automation is
possible to a certain extent ({\textbf{???}}), but human curation
remains necessary in the majority of cases (Jeangirard, Bracco, and
L'Hôte 2024). In this way, users retain control over the definition of
their perimeter, and can, if they wish, have several distinct
perimeters.

\hypertarget{code-availibility}{%
\section{4. Code availibility}\label{code-availibility}}

The code developed for the scanR web application is open source and
available online on GitHub \url{https://github.com/dataesr/scanr-ui}

\hypertarget{references}{%
\section*{References}\label{references}}
\addcontentsline{toc}{section}{References}

\hypertarget{refs}{}
\begin{cslreferences}
\leavevmode\hypertarget{ref-10.1162ux2fqss_a_00179}{}%
Chaignon, Lauranne, and Daniel Egret. 2022. ``Identifying Scientific
Publications Countrywide and Measuring Their Open Access: The Case of
the French Open Science Barometer (Bso).'' \emph{Quantitative Science
Studies} 3 (1): 18--36. \url{https://doi.org/10.1162/qss_a_00179}.

\leavevmode\hypertarget{ref-jeangirard:hal-04598201}{}%
Jeangirard, Eric, Laetitia Bracco, and Anne L'Hôte. 2024. ``Works-magnet
: aucune de perdue, 10 000 de retrouvées.'' Abes; Journées Abes 2024.
\url{https://doi.org/10.5281/zenodo.11471247}.

\leavevmode\hypertarget{ref-ipccbibliography}{}%
n.d.a. \url{https://www.ipcc.ch/report/ar6}.

\leavevmode\hypertarget{ref-ipbesbibliography}{}%
n.d.b.
\url{https://www.zotero.org/groups/2333077/ipbes_global_assessment/library}.
\end{cslreferences}


\end{document}
